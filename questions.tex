%%AMC:latex_engine=xelatex --shell-escape

\documentclass[11pt,french,a4paper,twoside]{article}

\usepackage{array,enumitem,graphicx,listings,multicol,multirow,titling,hyphenat,booktabs}

\usepackage[lang=FR,box,noshuffle,noshufflegroups,automarks,separateanswersheet]{automultiplechoice}
\usepackage{hyperref}
\title{Questions bidons}
\newcommand{\thesubtitle}{Examen blanc du \thedate{}}
\date{08/02/2024}
\author{François ROLAND}
\hypersetup{
  pdftitle={\thetitle},
  pdfauthor={\theauthor},
  pdflang={fr-BE},
  hidelinks}
\usepackage{fontspec}
\usepackage{libertine}
\usepackage{polyglossia}
\setdefaultlanguage{french}
\usepackage{csquotes}

\AMCrandomseed{1}
\AMCsetFoot{\thepage{}}

\begin{document}

%%% preparation of the groups

% chktex-file 19
\setdefaultgroupmode{withoutreplacement}

%%% questions (available snippets: mcsa[h|mc], mcma[h|mc], gra)
\element{q1}{
  \begin{question}{q1v1}
    Quelle est la couleur du cheval blanc d'Henri IV~?
    \begin{multicols}{2}
      \begin{choices}
        \correctchoice{blanc}
        \wrongchoice{noir}
        \wrongchoice{jaune}
        \wrongchoice{rouge}
        \lastchoices{}\columnbreak{}
        \wrongchoice{\textsc{aucune}}
        \wrongchoice{\textsc{toutes}}
        \wrongchoice{\textsc{manque}}
        \wrongchoice{\textsc{absurdité}}
      \end{choices}
    \end{multicols}
  \end{question}
}

%%% copies

\begin{examcopy}[2]
  \setcounter{figure}{0}
  %%% beginning of the header
  \noindent{\bf \thetitle{} \hfill{} \thesubtitle{}}

  \vspace{2ex}

  Durée~: 60~minutes.

  Aucun document n'est autorisé.
  La possession d'un objet électronique (smartphone, montre, écouteurs\ldots{}) pendant l'examen est interdite.

  Commencez toujours par indiquer lisiblement vos nom et prénom, ainsi que la date sur la feuille de réponse.

  Chaque question à choix multiple comporte \textbf{une} (et \textbf{une seule}) solution correcte.
  Certaines de ces solutions, appelées solutions générales, font appel à votre vigilance.
  \begin{itemize}
    \item \textsc{aucune} = aucune des solutions proposées n'est correcte.
    \item \textsc{toutes} = toutes les solutions proposées sont correctes.
    \item \textsc{manque} = il est impossible de répondre parce que l'information (au moins une donnée) manque dans l'énoncé de la question (donc pas dans le cours ni dans la connaissance actuelle du problème).
    \item \textsc{absurdité} = une absurdité dans l'énoncé rend toute la question sans objet (par exemple, une contrevérité dans l'énoncé).
  \end{itemize}
  Attention~! La réponse \textsc{absurdité} a priorité sur les autres solutions générales et, évidemment, sur les réponses spécifiques à la question.

  Une réponse correcte rapporte 1~point.
  Une réponse incorrecte, absente ou mal codée, rapporte 0~point.

  Vous pouvez inscrire ce que vous voulez sur le questionnaire et les feuilles de brouillon, mais seule la feuille de réponse sera corrigée.
  Vous devez rendre toutes les feuilles reçues, y compris celles de brouillon, à la fin de votre évaluation.

  \noindent{\hrulefill{}}

  \vspace{2ex}

  %%% end of the header

  \cleargroup{all}

  %%% copy question groups to the all group for randomization (available snippet: cgr)
  \copygroup[1]{q1}{all}
  \copygroup[1]{q2}{all}

  \insertgroup{all}

  \clearpage{}

  \begin{center}
    {\large{Feuille de brouillon}}
  \end{center}

  \AMCcleardoublepage{}

  \AMCformBegin{}

  \noindent{\begin{minipage}{.47\linewidth}
      {\bf \thetitle{} \\ \thesubtitle{}}

      {\large Feuille de réponse}
    \end{minipage}%
    \hfill{}%
    \namefield{\fbox{\begin{minipage}{.47\linewidth}%
          {\footnotesize Nom et prénom \par}%
          \vspace{\baselineskip}\namefielddots{}%
          \vspace*{1mm}%
        \end{minipage}}}}

  \vspace{2ex}

  Utilisez \textbf{uniquement} un \textbf{bic bleu ou noir}, jamais d'autre couleur, ni de crayon, ni de feutre.

  \textbf{Noircissez entièrement une seule case par question}.

  En cas de modification de votre réponse, utilisez toujours un ruban correcteur.
  Ne retracez jamais le contour d'une case ou sa lettre.
  N'entourez jamais de case.

  \noindent{\hrulefill{}}

  \vspace{2ex}
  %%% end of the answer sheet header

  \AMCform{}
  \AMCcleardoublepage{}

\end{examcopy}


\end{document}
